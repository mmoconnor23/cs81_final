\documentclass[11pt]{article}

\usepackage{fullpage}
\usepackage{graphicx}
\usepackage{caption}
\usepackage{subcaption}
\usepackage{float}

\title{Proposal: Predator, a New Motivation for Prey to Flock}
\author{Melissa O'Connor and Joe Boninger}
\date{April 7, 2014} %don't display the current date

\begin{document}
\maketitle

\section{Adaptive Method}
Joe
-NEAT - using existing software
-if time and promising results: potentially rtNEAT - our own implementation
        implementation plan: modify exiting NEAT code 

\section{The Learning Task}
Joe: environment: world of prey and a predator in a bounded ocean
Melissa: learning goal: generally, fitness based on density
               prey want to flock in order to scare away the predator
               predator wants to eat the prey but is scared of them if they
                are too close to each other
Joe: agent: simulated fish and shark in python simulator (not pyrobot)
Melissa: sensor inputs: 
        prey: distance from center of mass of the group of prey
              average direction the group is facing
              angle to turn to face the center of mass
              field of vision (grid system like predator's)
          Force the prey to move so they can't stop and "hide."
          Prey will be punished when the predator touches them.
        predator: density of fish in a grid system (4x4 grid) that 
                represents its field of vision
             The predator wants to move towards squares of low density (but
             not zero...he still wants to eat fish). 
             Punish if tries to eat his own predator.
          The predator can move faster than the prey so he has a chance of 
          catching them.
Melissa: motor outputs: translate and rotate
        prey: min speed above 0, max below predator's max
Joe: setup: 
        -give prey some time to learn how to flock without the presence of 
        the predator
        -predator also learning to avoid its own predator
                -will get confused when it sees a region of 0.9 fish and 
                also want to avoid that instead of eat it
        -if this predator method doesn't work, take away the predator's 
        predator and just punish for eating 0.9 density regions of prey
\section{Hypothesis and Outcome}
Melissa
Prey will flock! This will scare the predator so it doesn't eat them! Yay!
Predator will avoid the flocking prey because they look like his own 
predator and are scary!
\section{Analysis}
Melissa
1. visually observe the 'flock' 
2. quantitatively measure their density/concentration
3. count number of times predator tries to eat dense prey
4. average fitness and speciation graphs
        
\end{document}
